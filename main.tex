\documentclass[12pt]{article}

\title{Ficha Trigonometria}
\author{Tomás Pereira}

\begin{document}
\maketitle

1.\\\\
$\overline{DC}=1.7$\\
$tan\;77^{\circ}=\frac{\overline{DE}}{1.7}\equiv tan\;77^{\circ}\cdot1.7=\overline{DE}\equiv7.364\simeq\overline{DE}$\\
$\overline{AE}\simeq7.364+1.7\equiv\overline{AE}\simeq9.064$\\\\
R: A altura do monumento é aproximadamente $9$.\\

2.\\\\
$tan\;BAC=\frac{\overline{BC}}{\overline{BA}}$\\
$tan\;BAC=\frac{432}{564}\simeq0.765$\\
$tan^{-1}0.765\simeq37.416$\\\\
R: A amplitude do ângulo BAC é aproximadamente $37^{\circ}$.\\

3.\\\\
$\overline{AM}=\frac{AB}{2}$\\
$\overline{AM}=\frac{2.2}{2}$\\
$tan\;42^{\circ}=\frac{1.8}{\overline{MP}}\equiv\overline{MP}=\frac{1.8}{tan\;42^{\circ}}\equiv\overline{MP}\simeq1.999$\\
$\overline{MB}=\frac{2.2}{2}=1.1$\\
$\overline{BP}=1.999-1.1=0.899$\\\\
R: A Distância entre os pontos P e B é aproximadamente $0.9$.\\

4.\\\\
$cos\;26^{\circ}=\frac{10}{\overline{JG}}\equiv\overline{JG}\cdot cos\;26^{\circ}=10\equiv\overline{JG}\simeq11.126$\\
A[GHIJ] $\simeq16\cdot11.126\simeq178.016\;dm^2$\\\\
R: A área do painel fotovoltaico é aproximadamente $178\;dm^2$.\\

5.\\\\
$sen\;25^{\circ}=\frac{116}{\overline{FB}}\equiv\overline{FB}\cdot sen\;25^{\circ}=116\equiv\overline{FB}=\frac{116}{sen\;25^{\circ}}\equiv\overline{FB}\simeq274.479\;m$\\\\
R: O comprimento da rampa é aproximadamente $274\;m$.\\

6.\\\\
$sen\;A\hat{C}B=\frac{6}{7}\equiv sen\;A\hat{C}B\simeq0.857$\\
$A\hat{C}B\simeq sen^{-1}\;0.857\equiv A\hat{C}B\simeq58.981^{\circ}$\\\\
R: A amplitude do ângulo ACB é aproximadamente $59^{\circ}$.\\

7.\\\\
$\overline{AB}=8-0.16=7.84\;m$\\
$sen\;\alpha=\frac{7.84}{10.9}\equiv sen\;\alpha\simeq0.719$\\
$\alpha\simeq sen^{-1}\;0.719\equiv\alpha\simeq 45.972^{\circ}$\\\\
R: $\alpha$ é aproximadamente $46^{\circ}$.\\

8.\\\\
$sen\;\beta=\frac{\sqrt{5}}{3}$\\
$1=sen^2\;\beta+cos^2\;\beta$\\
$1=(\frac{\sqrt{5}}{3})^2+cos^2\;\beta\equiv1=\frac{5}{3^2}+cos^2\;\beta\equiv-cos^2\;\beta=\frac{5}{9}-1\equiv cos^2\;\beta=-(\frac{5}{9})+\frac{1}{1}\equiv cos^2\;\beta=\frac{-5}{9}+\frac{9}{9}\equiv cos^2\;\beta=\frac{4}{9}$\\
$cos\;\beta=\sqrt{\frac{4}{9}}\equiv cos\;\beta=\frac{2}{3}$\\

\end{document}
